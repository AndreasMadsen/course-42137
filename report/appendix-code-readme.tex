\section{Code ReadMe}

\subsubsection*{Requirements}

\begin{itemize}[noitemsep]
\item python 3.5 or 3.4
\item numpy (only used in gridsearch for storing results)
\item nose (only used in test for running tests)
\item tabulate (only used in plot for generating latex tables)
\end{itemize}

This install script will setup python 3 on the HPC cluster, but it is much
more complicated that what is needed for this project:
\url{https://github.com/AndreasMadsen/my-setup/tree/master/dtu-hpc-python3}

\subsubsection*{Examples}

\begin{itemize}[noitemsep]
\item \texttt{scripts/inspect\_tabu.py} will run a single optimization on the first dataset using TABU
\item \texttt{scripts/inspect\_alns.py} will run a single optimization on the first dataset using ALNS
\item \texttt{scripts/grid\_search\_tabu.py} the grid search script used for the TABU optimization
\item \texttt{scripts/grid\_search\_alns.py} the grid search script used for the ALNS optimization
\item \texttt{scripts/test\_tabu.py} the test script used for the TABU optimization
\item \texttt{scripts/test\_alns.py} the test script used for the ALNS optimization
\end{itemize}

\subsubsection*{Directories}

The code is structured into the following directories:

\begin{itemize}[noitemsep]
\item \texttt{dataset}: contains \texttt{Database} constructor, used for initializing a dataset
\item \texttt{gridsearch}: code for running many optimizations using different parameters
\item \texttt{judge}: code for validating the solution using \texttt{Judge.exe}.
\item \texttt{plot}: code for generating latex tables and other summaries
\item \texttt{script}: scripts used for manual inspecting and running on the HPC cluster
\item \texttt{search}: the ALNS and TABU implementations are in here
\item \texttt{solution}: contains \texttt{Solution} constructor that contains the an solution. Its
 methods simulate moves or mutate the solution. It also contains the random
 initialization procedure.
\item \texttt{test}: test scripts, run them using \texttt{make test}
\end{itemize}
