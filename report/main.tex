\documentclass[a4paper]{article}

\usepackage[utf8]{inputenc}	% Flere sprog tegnsæt (fx æøå)
\usepackage[english]{babel}	% Dansk orddeling (kan ændres til english)
\usepackage[T1]{fontenc}		% Brug 8-bit front
\usepackage{lmodern}		% Vektor front

\usepackage[svgnames]{xcolor} % Udvider \color med "SVG color names"
\usepackage{graphicx}	% Kompatibilitet til visning af pixel billeder (.png, .jpg, .gif)
\usepackage{epstopdf}	% Kompatibilitet til visning af vector billeder (.eps)
\usepackage{parskip}	% Tilføjer vertikal margin til hver paragraph
\usepackage{float}		% TIllader H som positions parameter
\usepackage{subcaption}	% Tillader subfigure, subtable samt \captions
\usepackage{multirow,tabularx}
\usepackage{amssymb}	% Flere matematiske symboler
\usepackage{amsthm}      % Endnu flere matematiske symboler
\usepackage{mathtools}	% Det meste matematik (indeholder ams­math og rettelser)
\usepackage{xfrac}		% Flere fracs (\sfrac{}{})
\usepackage{listings}	% Indsæt code
\usepackage{algorithm}
\usepackage[noend]{algpseudocode}
\usepackage{fancyhdr}	% Side hoved og sidefod
\usepackage{todonotes}	% Cool to-do notes, [disable] skjuler to-do
\usepackage[bookmarks,bookmarksnumbered,hidelinks]{hyperref} % clickable pdf (til sidst)
\usepackage[bibstyle=ieee,citestyle=numeric-comp]{biblatex} % Benyt BibLaTeX til formatering

%listing settings, æøå support, font config, line number, left lines
\lstset{
    breakatwhitespace=false, breaklines=true,
    inputencoding=utf8, extendedchars=true,
    literate={å}{{\aa}}1 {æ}{{\ae}}1 {ø}{{\o}}1 {Å}{{\AA}}1 {Æ}{{\AE}}1 {Ø}{{\O}}1,
    keepspaces=true, showstringspaces=false, basicstyle=\small\ttfamily,
    frame=L, numbers=left, numberstyle=\scriptsize\color{gray},
    keywordstyle=\color{SteelBlue}\ttfamily,
    stringstyle=\color{IndianRed}\ttfamily,
    commentstyle=\color{Teal}\ttfamily,
    captionpos=b,
}

% algorithm environment
%http://tex.stackexchange.com/questions/1375/what-is-a-good-package-for-displaying-algorithms
\algnewcommand{\Let}[2]{\State #1 $\gets$ #2}
\algnewcommand{\Not}[0]{\textbf{not }}
\algnewcommand{\Implicit}[1]{\State \textit{#1}}
\algnewcommand{\LineComment}[1]{\State \(\triangleright\) #1}
\algrenewcommand\Call[2]{\textproc{#1}(#2)}
\algrenewcommand\alglinenumber[1]{{\footnotesize\color{gray}\ttfamily#1}}

\setlength{\marginparwidth}{80pt} 				% Mere brede på margin notes og to-do
\setlength{\parindent}{0cm}   					% Deaktiver afsnit indrykning
\DeclareGraphicsExtensions{.pdf,.eps,.png,.jpg,.gif}	% ændre til .png, .jpg for hurtig visning
\pagestyle{fancy}
\fancyhead[L]{Andreas Madsen - s123598}

\setcounter{tocdepth}{2}
\numberwithin{equation}{section}

\addbibresource{bibliography.bib}

\begin{document}

\title{University Timetabling \\ {\large Curriculum-based Course Timetabling using ALNS and TABU}}
\author{Andreas Madsen – s123598}
\date{April 25. 2016}

\maketitle
\tableofcontents
\pagebreak

\section{Introduction}

The curriculum-based university timetabling problem is about assigning courses, which are associated to a set of curriculums, to a schedule. Each slot in this schedule consists of a room and a time, where the time is discretized into a given number of days and periods per day. Each course are to be scheduled a given number of times.

The curriculum part of the problem, is about making sure a student who attend a given curriculum can take all courses, meaning there are no overlapping courses within the  curriculum. There is also an optimization component, which prefers schedule where there are no holes for a given curriculum during a day (called curriculum compactness).

For solving this problem a meta heuristic approach is taken. Tabu Search and ALNS will both be used to solve the problem independently and then be compared.
\section{Problem Description}

The curriculum-based university timetabling problem is about assigning courses, which are associated to a set of curriculums, to a schedule. Each slot in this schedule consists of a room and a time, where the time is discretized into a given number of days and periods pr. day. Each course are to be scheduled a given number of times.

The curriculum part of the problem, is about making sure a student who attend a given curriculum can take all courses, meaning there are no overlapping courses within the  curriculum. There is also an optimization component, which prefers schedule where there are no holes for a given curriculum (called curriculum compactness).

The problem is very precisely defined in the project description \cite{assignment}. The problem will be restated here, but also discussed in terms of the the delta calculations can be done.

Delta calculations is the change in the objective value when performing a simple operation. In this case only adding or removing a single course from a slot (room and time) is considered. In section \ref{sec:mutate-operations} these operations will be precisely defined and it will be discussed how other operations can be implemented using these primitives.

The used notation is the same as in the project description \cite{assignment} and may also be found in appendix \ref{appendix:notation}.

\subsection{Constraints}

It will always be valid to remove a course, thus the following constraints only need to be checked when adding a course to the solution schedule.

\subsubsection{Time availability}
\begin{equation}
    \sum_{r \in R} x_{c, t, r} \le F_{c, t} \quad \forall c \in C, t \in T
\end{equation}

A course may only be assigned once in the same time slot. The dataset also contains a set of unavailable time slots for each course. For those time slots $F_{c, t} = 0$ otherwise $F_{c, t} = 1$.

\textbf{Delta:} This constraint is really two separate constraints. 1. Maintain an indicator that is true if the given course is assigned to the given time. Secondly one can use a lookup table to check if the time slot is defined unavailable.

\subsubsection{Room availability}
\begin{equation}
    \sum_{c \in C} x_{c, t, r} \le 1 \quad \forall t \in T, r \in R
\end{equation}

A slot consisting of room and time may only be used once.

\textbf{Delta:} Maintain an indicator that is true if the slot (time and room) is used.

\subsubsection{Max lectures}
\begin{equation}
\sum_{t \in T, r \in R} x_{c, t, r} \le L_c \quad \forall c \in C
\end{equation}

Prevents more slots to a course than what is needed.

\textbf{Delta: } Maintain a counter of how many times a given course is assigned.

\subsubsection{Conflicting courses}
\begin{equation}
\sum_{r \in R} x_{c_1, t, r} + \sum_{r \in R} x_{c_2, t, r} \le 1 \quad \forall c_1, c_2 \in C, t \in T, \chi(c_1, c_2) = 1
\end{equation}

Conflicting courses may not be assigned the same time slot. Courses are conflicting is taught by the same lecture or part of the same curriculum.

\textbf{Delta: } Pre-calculate a list of of conflicting courses for each course and maintain a list of courses assigned to each time slot. Then check that there are no conflicting courses assigned to the given time slot.

\subsection{Objective}

There are 5 objectives which are combined to a single objective score:
\begin{equation}
\mathrm{Objective} = 10 U + 5 W + 2 A + 1 P + 1 V
\end{equation}

The delta calculationse for the add and remove operations are mostly part the same, just with the sign flipped. For $P$ and $W$ there are minor differences. Furthermore it can be assumed that the constrains are met.

\subsubsection{Room capacity (V)}
\begin{equation}
V = \sum_{t \in T, r \in R} V_{t, r}(x), \quad V_{t, r} = \mathrm{max}\left\{ 0, \sum_{c \in C} S_c \cdot x_{c, t, r} - C_r \right\}
\end{equation}

A course has an associated number of students $S_c$ and a room has an associated capacity $C_r$. $V$ is the penalty for exceeding this capacity.

\textbf{Delta: } One can simply lookup $S_c$ and $C_r$ for a given $(course, time, room)$ combination.

\subsubsection{Unscheduled (U)}
\begin{equation}
U = \sum_{c \in C} U_c(x), \quad U_c(x) = \mathrm{max}\left\{ 0, L_c - \sum_{t \in T, r \in R} x_{c, t, r}\right\}
\end{equation}

A course is to be scheduled a given number of times $L_c$.

\textbf{Delta: } Since the new solution is valid, one can simply subtract or add one to the $U$ sum.

\subsubsection{Room stability (P)}
\begin{equation}
P = \sum_{c \in C} P_c(x), \quad P_c(x) = \mathrm{max}\left\{ 0, \left|\left| \left\{ r \in R \Big\vert \sum_{t \in T} x_{c, t, r} \ge 1 \right\} \right|\right| - 1 \right\}
\end{equation}

A course should be assigned to as few different rooms as possible.

\textbf{Delta: } On addition check if $r$ is a new room for the course. On remove check if room $r$ is used only once in the current solution.

\subsubsection{Minimum working days (W)}
\begin{equation*}
W = \sum_{c \in C} W_c(x), \quad W_c(x) = \mathrm{max}\left\{ 0, M_c - \left|\left| \left\{ d \in D \Big\vert \sum_{t \in T(d), r \in R} x_{c, t, r} \ge 1 \right\} \right|\right| \right\}
\end{equation*}

A course should be assigned times over $M_c$ different days.

\textbf{Delta: } Count the number of days the course is spread over. On addition check if this count is less than $M_c$ and check that the course isn't already scheduled this day. On remove check that the count isn't greater than $M_c$ and that the course isn't scheduled times during this day.

\subsubsection{Curriculum compactness (A)}
\begin{equation*}
A = \sum_{q \in Q, t \in T} A_{q, t}(x), \quad A_{q, t}(x) = \begin{cases}
1 & \text{if} \displaystyle \sum_{c \in C(q), r \in R} x_{c, t, r} = 1 \wedge \sum_{\substack{c \in C(q), r \in R,\\ t' \in T, \Upsilon(t, t')  = 1}} x_{c, t', r} = 0 \\
0 & \text{otherwise}
\end{cases}
\end{equation*}

Penalizes gaps/holes over a day in a curriculum.

\textbf{Delta: } This is the most complicated objective in terms of delta calculation, because one also need to adjust the penalty for adjacent courses in the same curriculum. Thus if the course before was penalized then remove penalty, similarly for the course after. If there are no courses before and after add a penalty.

\subsection{Datasets}

There are 12 datasets each is given as 7 files:
\begin{itemize}
\item \texttt{basic.utt} - Contains basic meta information about the problem. Number of courses, rooms, days, periods pr day, etc.
\item \texttt{courses.utt} - Contains information about each course.
\item \texttt{curricula.utt} - Contains the number of courses associated to each curriculum. This file was not used as the information could be inferred from \texttt{relation.utt}.
\item \texttt{lecturers.utt} - Contains a list of lecturers. This file was not used as the information could be inferred from \texttt{courses.utt}.
\item \texttt{relation.utt} - Relational table that binds curriculum and course.
\item \texttt{rooms.utt} - Contains information about each room.
\item \texttt{unavailability.utt} - Contains the list of unavailable time slots for each course.
\end{itemize}

Using \texttt{basic.utt} the 12 datasets can be summarized:


\begin{table}[H]
\centering
\centerline{
\begin{tabular}{r|c c c c c c c}
	dataset & Courses & Rooms & Days & Periods per day & Curricula & Constraints & Lecturers\\ \hline
	1 & 30 & 6 & 5 & 6 & 14 & 53 & 24 \\
	2 & 82 & 16 & 5 & 5 & 70 & 513 & 71 \\
	3 & 72 & 16 & 5 & 5 & 68 & 382 & 61 \\
	4 & 79 & 18 & 5 & 5 & 57 & 396 & 70 \\
	5 & 54 & 9 & 6 & 6 & 139 & 771 & 47 \\
	6 & 108 & 18 & 5 & 5 & 70 & 632 & 87 \\
	7 & 131 & 20 & 5 & 5 & 77 & 667 & 99 \\
	8 & 86 & 18 & 5 & 5 & 61 & 478 & 76 \\
	9 & 76 & 18 & 5 & 5 & 75 & 405 & 68 \\
	10 & 115 & 18 & 5 & 5 & 67 & 694 & 88 \\
	11 & 30 & 5 & 5 & 9 & 13 & 94 & 24 \\
	12 & 88 & 11 & 6 & 6 & 150 & 1368 & 74
\end{tabular}}
\caption{Information from \texttt{basic.utt} for each dataset.}
\end{table}


\subsection{Choosing the metaheuristic models}

As seen the problem is highly constrained problem. ALNS is usually good for highly constrained problems, thus ALNS is an obvious choices.

The objective function is not multi objective, but do consist of many separate sub-objective functions. This suggest Evolutionary Algorithms may be a good choice, as two good solution may minimize two different sub-objectives, thus a crossover between the two solutions could minimize both and become a very good solution. However because of the tightly constrained nature of the problem, it will be difficult to come up with a good crossover algorithm.

GRAPS is likely a poor choice, as the many constraints makes it difficult to come up with many good greedy algorithms.

The neighborhood of a given solution is at least, all possible $(course, time, room)$ additions and removal operations. However because of the big penalty (+10) of removing a $(course, time, room)$ combination, this is unlikely to ever be an immediate good choice. Thus it could makes sense to expand the neighborhood to include all possible moves to an available slot and all possible swaps between two $(course, time, room)$ combinations. The expanded neighborhood is a huge, thus ALNS may also be a good choice.

\subsubsection{Conclustion}

ALNS is chosen because of the many constraints. TABU is chosen because of the large neighborhood. Evolutionary Algorithms is an interesting choice and may be worth investigating, however that is out of the scope for this project.
 % Lør
\section{Metaheuristic description}
\subsection{Mutatation operations}
\label{sec:mutate-operations}

The solution schedule is defined as list of $(c \in C, t \in T, r \in R)$, that is course $c$ is assigned to a slot given by the time $t$ and room $r$.

In this project 4 move operations are used. The operations are:
\begin{itemize}
\item \texttt{Add($c, t, r$)} - adds course $c$ to the slot $(t, r)$.  
\item \texttt{Remove($c, t, r$)} - remove course $c$ from the schedule slot $(t, r)$.
\item \texttt{Move($c, t_0, r_0, t_1, r_1$)} - moves course $c$ from $(t_0, r_0)$ to $(t_1, r_1)$.
\item \texttt{Swap($c_0, t_0, r_0, c_1, t_1, r_1$)} - course $c_0$ and course $c_1$ swaps slots.
\end{itemize}

To avoid copying the solution object the above cost of the operations, can be calculated without actually changing the solution object. Those functions are prefixed with \texttt{Simulate}, while the functions that actually changes the solution object are prefixed with \texttt{Mutate}.

\begin{algorithm}[H]
  \caption{Add a course $c$ to slot $(t, r)$}
  \begin{algorithmic}[1]
    \Function{SimulateAdd}{$c, t, r$}
      \If{\Not \Call{ValidAdd}{$c, t, r$}} \Comment{Discussed in section \ref{sec:problem-constraints}}
          \State \Return{None}
      \EndIf
      \State \Return{\Call{CostAdd}{$c, t, r$}} \Comment{Discussed in section \ref{sec:problem-objective}}
    \EndFunction
    
    \Statex
    \Function{MutateAdd}{$c, t, r$}
        \Let{$\Delta$}{\Call{SimulateAdd}{$c, t, r$}}
        \If{$\Delta \not= None$}
            \Let{$Objective$}{$Objective + \Delta$}
            \Implicit{update data structures} \Comment{Discussed in section \ref{sec:problem-constraints} and \ref{sec:problem-objective}}
        \EndIf
    \EndFunction
  \end{algorithmic}
\end{algorithm}


\begin{algorithm}[H]
  \caption{Remove a course $c$ from slot $(t, r)$}
  \begin{algorithmic}[1]
    \Function{SimulateRemove}{$c, t, r$}
      \If{\Not \Call{ValidRemove}{$c, t, r$}} \Comment{Discussed in section \ref{sec:problem-constraints}}
          \State \Return{None}
      \EndIf
      \State \Return{\Call{CostRemove}{$c, t, r$}} \Comment{Discussed in section \ref{sec:problem-objective}}
    \EndFunction
    
    \Statex
    \Function{MutateRemove}{$c, t, r$}
        \Let{$\Delta$}{\Call{SimulateRemove}{$c, t, r$}}
        \If{$\Delta \not= None$}
            \Let{$Objective$}{$Objective + \Delta$}
            \Implicit{update data structures} \Comment{Discussed in section \ref{sec:problem-constraints} and \ref{sec:problem-objective}}
        \EndIf
    \EndFunction
  \end{algorithmic}
\end{algorithm}

Note that one typically simulates the operation to check that $\Delta < 0$ and then mutate the solution object. Thus it makes sense to add $\Delta$ as an optional argument to the mutate functions, to avoid unnecessary calculations. This was done in the implementation but is for simplicity excluded here.

The \texttt{Move} and \texttt{Swap} operations can then be implemented, using using the \texttt{Add} and \texttt{Remove} primitives. This is not the most efficient implementation as it requires mutation of the current solution, which are then reverted in the simulation case. However given the time constraints, the more efficient implementations was made. See appendix \ref{appendix:operations} for how \texttt{Move} and \texttt{Swap} was implemented using the primitives.

\input{gready-initialization}
\subsection{TABU description}

\subsection{TABU specialization}
\subsection{ALNS description}

\subsection{ALNS specialization}
 % Søn
\section{Parameter tuning}

In order to find the best set of parameters for the ALNS and Tabu search, different parameter combinations was tried (see section \ref{sec:parameter-tabu} and \ref{sec:parameter-alns}). Each parameter combination was tried 3 times using different initializations.

Because the problems aren't equally difficult and because the objective value aren't normalized, the objective value for each dataset can't be directly compared. To accommodate the best objective value for each dataset is used to normalize the objective (percentage gap):
\begin{equation}
\tilde{z}_i = \frac{z_i - z^*}{z^*}
\end{equation}
Here $z_i$ is the objective value and $z^*$ is the best objective value for the dataset.

Because one wishes to avoid overfitting of the parameters, a subset of the entire dataset is chosen for parameter optimization, this is called the training dataset. As there do not appear to be any pattern in the dataset id, all odd dataset are chosen for parameter optimization.

\begin{table}[H]
\centering
\begin{tabular}{l|rrrrrrr}
 dataset id &   1 &   3 &   5 &    7 &   9 &   11 &   13 \\
\hline
 Tabu   &  35 & 706 & 896 & 1390 & 765 &   36 &  794 \\
 ALNS   &  24 & 211 & 761 &  211 & 200 &    5 &  166 \\
 both   &  24 & 211 & 761 &  211 & 200 &    5 &  166 \\
\end{tabular}
\caption{Best objective value for each training dataset}
\end{table}

\subsection{Tabu}
\label{sec:parameter-tabu}

\begin{table}[H]
\centering
\centerline{\begin{tabular}{rr|ccc}
 &  & \multicolumn{3}{c}{\texttt{intensification}}\\
 &  & 2 & 10 & None\\
\hline
\multirow{3}{*}{\texttt{diversification}} & None & (4.90, 0.52) & (5.36, 0.37) & (5.53, 0.71)\\
 & 1 & (5.07, 0.97) & (5.34, 0.17) & (5.29, 0.63)\\
 & 5 & (5.19, 0.41) & (4.14, 0.12) & (5.06, 0.77)\\
\end{tabular}}
\caption{Shows $(\mu, \sigma)$ with \texttt{allow\_swap=dynamic} and \texttt{tabu\_limit=40} fixed}
\end{table}

\begin{table}[H]
\centering
\centerline{\begin{tabular}{rr|cccc}
 &  & \multicolumn{4}{c}{\texttt{tabu\_limit}}\\
 &  & 10 & 20 & 40 & None\\
\hline
\multirow{3}{*}{\texttt{allow\_swap}} & never & (5.53, 0.70) & (5.91, 0.89) & (6.52, 0.60) & (5.93, 0.43)\\
 & always & (9.12, 0.18) & (8.77, 0.26) & (9.16, 0.87) & (8.82, 0.44)\\
 & dynamic & (5.52, 0.69) & (5.10, 0.59) & (4.14, 0.12) & (5.45, 0.27)\\
\end{tabular}}
\caption{Shows $(\mu, \sigma)$ with \texttt{diversification=5} and \texttt{intensification=10} fixed}
\end{table}

\begin{table}[H]
\centering
\begin{tabular}{r|c}
parameter & value \\ \hline
allow swap & dynamic \\
tabu limit & 40 \\
intensification & 10 \\
diversification & 5
\end{tabular}
\caption{Best Tabu search parameters with $\mu = 4.139$ and $\sigma = 0.122$}
\end{table}

\subsection{ALNS}
\label{sec:parameter-alns}

\begin{table}[H]
\centering
\centerline{\begin{tabular}{rr|ccc}
 &  & \multicolumn{3}{c}{\texttt{remove}}\\
 &  & 1 & 3 & 5\\
\hline
\multirow{3}{*}{\texttt{update\_lambda}} & 0.9 & (0.50, 0.10) & (0.59, 0.01) & (0.82, 0.05)\\
 & 0.95 & (0.30, 0.08) & (1.07, 0.03) & (1.02, 0.12)\\
 & 0.99 & (0.56, 0.09) & (1.69, 0.11) & (1.52, 0.07)\\
\end{tabular}}
\caption{Shows $(\mu, \sigma)$ with \texttt{w\_global=10} and \texttt{w\_current=5} fixed}
\end{table}

\begin{table}[H]
\centering
\centerline{\begin{tabular}{rr|ccc}
 &  & \multicolumn{3}{c}{\texttt{w\_current}}\\
 &  & 1 & 3 & 5\\
\hline
\multirow{2}{*}{\texttt{w\_global}} & 5 & (0.55, 0.06) & (0.52, 0.12) & (0.44, 0.03)\\
 & 10 & (0.49, 0.04) & (0.32, 0.10) & (0.30, 0.08)\\
\end{tabular}}
\caption{Shows $(\mu, \sigma)$ with \texttt{update\_lambda=0.95} and \texttt{remove=1} fixed}
\end{table}

\begin{table}[H]
\centering
\begin{tabular}{r|c}
parameter & value \\ \hline
$\lambda$ & 0.95 \\
$w_{global}$ & 10 \\
$w_{current}$ & 5 \\
remove & 1 \\
\end{tabular}
\caption{Best ALNS parameters with $\mu = 0.3024$ and $\sigma = 0.0846$}
\end{table} % Man
\section{Test results}

The best parameters form the parameter tuning in section \ref{sec:parameter-tuning}, are used to test both the Tabu and ALNS model. All odd numbered datasets was used for parameter tuning, thus those datasets shouldn't be used for testing. But for completeness the search algorithms was reapplied on the training set. This time the parameters are fixed thus the number of runs can be increased to 5 runs per dataset.

\begin{table}[H]
\centering
\centerline{\begin{tabular}{rr|cc}
& & Tabu & ALNS \\
\hline
\multirow{7}{*}{train} & 1 & (3.00, 0.84) & (0.22, 0.18) \\
& 3 & (3.07, 0.08) & (0.23, 0.15) \\
& 5 & (0.14, 0.04) & (0.07, 0.05) \\
& 7 & (6.33, 0.22) & (0.20, 0.11) \\
& 9 & (3.19, 0.24) & (0.13, 0.09) \\
& 11 & (12.80, 2.45) & (0.80, 0.81) \\
& 13 & (3.90, 0.17) & (0.11, 0.11) \\
\hline
\multirow{6}{*}{test} & 2 & (3.17, 0.27) & (0.09, 0.11) \\
& 4 & (4.88, 0.35) & (0.09, 0.08) \\
& 6 & (5.06, 0.35) & (0.05, 0.05) \\
& 8 & (5.19, 0.43) & (0.12, 0.07) \\
& 10 & (6.22, 0.39) & (0.14, 0.07) \\
& 12 & (0.92, 0.10) & (0.06, 0.06) \\
\hline
\multicolumn{2}{c|}{all train} & (4.63, 0.46) & (0.25, 0.14) \\
\multicolumn{2}{c|}{all test} & (4.24, 0.15) & (0.09, 0.04) \\
\end{tabular}}
\caption{Test and train results over 5 runs using best parameters}
\end{table}

The Tabu mean is a bit too high given the standard deviance calculated in section \ref{sec:parameter-tuning}. This appears to be caused by dataset number 11, where the Tabu search performs extremely poorly. But overall the values are close to what one would expect, from the parameter tuning in section \ref{sec:parameter-tuning}. The test objective values are actually surprisingly good, as they are a little smaller than those produced by the train datasets.
 % Man, Fre
\section{Conclusion}

ALNS consistently outperforms Tabu Search. As both algorithms reaches a point where it becomes hard to find a much better solution, this must be because ALNS covers a much broader solution space.

Diversification in Tabu Search is meant to move the search to another part of the solution space. But clearly this is not accomplished sufficiently. This is likely because Tabu Search spends a lot of time just in validating solutions, this is particularly the case with the swap operation. ALNS don't do this, as it just looks at the best placement for missing courses. Improving the performance of the move and swap operations could make a huge difference for Tabu Search. The fact that the dynamic neighborhood consistently outperforms any other neighborhood credits this hypothesis too.

Even if the move and swap operations where made faster, it is possible that the complexity of the problem makes it impossible for those operations to become sufficiently fast. Using a dynamic neighborhood thus still makes sense. In particularly this is something that could be applied to many other problems. It has definitely been shown to make a big improvement.

In parameter tuning for ALNS it was found that using the same $w_{current}$ and $w_{global}$ gave the best results. This is something that is not intuitive at first, but as discussed indicates that a globally better solution is found by a long line of good repair and destroy choices. If one did not have the computational resources to optimize these parameters, setting them to the same value may be a good choice.

To improve the ALNS method further, one could analyze the repair and destroy method selection probabilities to learn about what works and more importantly doesn't work for some datasets. Using this knowledge more fine tuned algorithms could be constructed. However one also have to be careful not to do such must fine tuning that it becomes overfitting.

At last one could explore combining the methods. A simple way of doing this, could be to initialize the Tabu Search using the ALNS solution. This would ensure that the ALNS solution is in a local minima and there isn't some nearby better solution, that can be found by simple diversification.
 % Fre

\pagebreak
\appendix
\section{Notaton}
\label{appendix:notation}

Notation and text descriptions are from the project description \cite{assignment}.

\subsubsection*{Sets}

\begin{description}
\item[$C$] The set of courses
\item[$L$] The set of lecturers
\item[$R$] The set of rooms
\item[$Q$] The set of curricula
\item[$T$] The set of time slots. i.e. all pairs of days and periods
\item[$D$] The set of days
\item[$T(d)$] The set of time slots that belongs to day $d \in D$
\item[$C(q)$] The set of courses that belongs to curriculum $q \in Q$
\end{description}

\subsubsection*{Parameters}

\begin{description}
\item[$L_c$] The number of lectures there should be for course $c \in C$
\item[$C_r$] The capacity of room $r \in R$
\item[$S_c$] The number of students attending course $c$
\item[$M_c$] The minimum number of days that course c should be spread across
\end{description}

\begin{align*}
F_{c,t} &= \begin{cases}
1 & \text{if course $c \in C$ is available in time slot $t \in T$} \\
0 & \text{otherwise}
\end{cases} \\
\chi(c_1, c_2) &= \begin{cases}
1 & \parbox[t]{.8\textwidth}{if course $c_1 \in C$ is different from course $c_2 \in C$ ($c_1 \not = c_2$) and conflicting, ie. are taught by the same lecturer or are part of the same curriculum.} \\
0 & \text{otherwise}
\end{cases} \\
\Upsilon(t_1, t_2) &= \begin{cases}
1 & \parbox[t]{.8\textwidth}{if time slot $t_1$ and $t_2$ belongs to the same day and are adjacent to each other} \\
0 & \text{otherwise}
\end{cases}
\end{align*}

\subsubsection*{Decision Variables}

\begin{equation*}
x_{c, t, r} = \begin{cases}
1 & \text{if class $c \in C$ is allocated to room $r \in R$ in time slot $t\in T$} \\
0 & \text{otherwise}
\end{cases}
\end{equation*}

\section{Move and swap operation}
\label{appendix:operations}


\begin{algorithm}[H]
  \caption{Move a course $c$ from slot $(t_0, r_0)$ to $(t_1, t_2)$}
  \begin{algorithmic}[1]
    \Function{SimulateMove}{$c, t_0, r_0, t_1, r_1$}
      \Let{$\Delta_{remove}$}{\Call{SimulateRemove}{$c, t_0, r_0$}}
      \If{$\Delta_{remove} = None$}
          \State \Return{None}
      \EndIf
      \State \Call{MutateRemove}{$c, t_0, r_0$}
      \State
      \Let{$\Delta_{add}$}{\Call{SimulateAdd}{$c, t_1, r_1$}}
      \If{$\Delta_{add} = None$}
          \State \Call{MutateAdd}{$c, t_0, r_0$} \Comment{Revert remove operation}
          \State \Return{None}
      \EndIf
      \State
      \State \Call{MutateAdd}{$c, t_0, r_0$} \Comment{Revert remove operation}
      \State
      \State\Return{$\Delta_{remove} + \Delta_{add}$}
    \EndFunction
    
    \Statex
    \Function{MutateMove}{$c, t_0, r_0, t_1, r_1$}
        \Let{$\Delta$}{\Call{SimulateMove}{$c, t_0, r_0, t_1, r_1$}}
        \If{$\Delta \not= None$}
            \State\Call{MutateRemove}{$c, t_0, r_0$}
            \State\Call{MutateAdd}{$c, t_1, r_1$}
        \EndIf
    \EndFunction
  \end{algorithmic}
\end{algorithm}


\begin{algorithm}[H]
  \caption{Course $c_0$ and course $c_1$ swaps slots.}
  \begin{algorithmic}[1]
    \Function{SimulateSwap}{$c_0, t_0, r_0, c_1, t_1, r_1$}
      \Let{$\Delta_{remove, 0}$}{\Call{SimulateRemove}{$c_0, t_0, r_0$}}
      \If{$\Delta_{remove, 0} = None$}
          \State \Return{None}
      \EndIf
      \State \Call{MutateRemove}{$c_0, t_0, r_0$}
      \State
      \Let{$\Delta_{remove, 1}$}{\Call{SimulateRemove}{$c_1, t_1, r_1$}}
      \If{$\Delta_{remove, 1} = None$}
          \State \Call{MutateAdd}{$c_0, t_0, r_0$} \Comment{Revert remove operation}
          \State \Return{None}
      \EndIf
      \State \Call{MutateRemove}{$c_1, t_1, r_1$}
      \State
      \Let{$\Delta_{add, 0}$}{\Call{SimulateAdd}{$c_0, t_1, r_1$}}
      \If{$\Delta_{add, 0} = None$}
          \State \Call{MutateAdd}{$c_1, t_1, r_1$} \Comment{Revert operations}
          \State \Call{MutateAdd}{$c_0, t_0, r_0$}
          \State \Return{None}
      \EndIf
      \State \Call{MutateAdd}{$c_0, t_1, r_1$}
      \State
      \Let{$\Delta_{add, 1}$}{\Call{SimulateAdd}{$c_1, t_0, r_0$}}
      \If{$\Delta_{add, 1} = None$}
          \State \Call{MutateRemove}{$c_0, t_1, r_1$} \Comment{Revert operations}
          \State \Call{MutateAdd}{$c_1, t_1, r_1$}
          \State \Call{MutateAdd}{$c_0, t_0, r_0$}
          \State \Return{None}
      \EndIf
      \State
      \State \Call{MutateRemove}{$c_0, t_1, r_1$} \Comment{Revert operations}
      \State \Call{MutateAdd}{$c_1, t_1, r_1$}
      \State \Call{MutateAdd}{$c_0, t_0, r_0$}
      \State
      \State \Return{$\Delta_{remove, 0} + \Delta_{remove, 1} + \Delta_{add, 0} + \Delta_{add, 1}$}
    \EndFunction
    
    \Statex
    \Function{MutateSwap}{$c_0, t_0, r_0, c_1, t_1, r_1$}
        \Let{$\Delta$}{\Call{SimulateSwap}{$c_0, t_0, r_0, c_1, t_1, r_1$}}
        \If{$\Delta \not= None$}
            \State \Call{MutateRemove}{$c_0, t_0, r_0$}
            \State \Call{MutateRemove}{$c_1, t_1, r_1$}
            \State \Call{MutateAdd}{$c_0, t_1, r_1$}
            \State \Call{MutateAdd}{$c_1, t_0, r_0$}
        \EndIf
    \EndFunction
  \end{algorithmic}
\end{algorithm}

\section{Code ReadMe}

\subsubsection*{Requirements}

\begin{itemize}[noitemsep]
\item python 3.5 or 3.4
\item numpy (only used in gridsearch for storing results)
\item nose (only used in test for running tests)
\item tabulate (only used in plot for generating latex tables)
\end{itemize}

This install script will setup python 3 on the HPC cluster, but it is much
more complicated that what is needed for this project:
\url{https://github.com/AndreasMadsen/my-setup/tree/master/dtu-hpc-python3}

\subsubsection*{Running the solver}

Only python 3.4 or python 3.5 is required for running the solver.

\begin{lstlisting}
python3 ./solver.py courses.utt lecturers.utt rooms.utt curricula.utt relation.utt unavailability.utt 300
\end{lstlisting}

\subsubsection*{Examples}

\begin{itemize}[noitemsep]
\item \texttt{scripts/inspect\_tabu.py} will run a single optimization on the first dataset using TABU
\item \texttt{scripts/inspect\_alns.py} will run a single optimization on the first dataset using ALNS
\item \texttt{scripts/grid\_search\_tabu.py} the grid search script used for the TABU optimization
\item \texttt{scripts/grid\_search\_alns.py} the grid search script used for the ALNS optimization
\item \texttt{scripts/test\_tabu.py} the test script used for the TABU optimization
\item \texttt{scripts/test\_alns.py} the test script used for the ALNS optimization
\end{itemize}

\subsubsection*{Directories}

The code is structured into the following directories:

\begin{itemize}[noitemsep]
\item \texttt{dataset}: contains \texttt{Database} constructor, used for initializing a dataset
\item \texttt{gridsearch}: code for running many optimizations using different parameters
\item \texttt{judge}: code for validating the solution using \texttt{Judge.exe}.
\item \texttt{plot}: code for generating latex tables and other summaries
\item \texttt{script}: scripts used for manual inspecting and running on the HPC cluster
\item \texttt{search}: the ALNS and TABU implementations are in here
\item \texttt{solution}: contains \texttt{Solution} constructor that contains the an solution. Its
 methods simulate moves or mutate the solution. It also contains the random
 initialization procedure.
\item \texttt{test}: test scripts, run them using \texttt{make test}
\end{itemize}


\pagebreak
\printbibliography

\end{document}
