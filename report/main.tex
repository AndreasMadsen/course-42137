\documentclass[a4paper]{article}

\usepackage[utf8]{inputenc}	% Flere sprog tegnsæt (fx æøå)
\usepackage[english]{babel}	% Dansk orddeling (kan ændres til english)
\usepackage[T1]{fontenc}		% Brug 8-bit front
\usepackage{lmodern}		% Vektor front

\usepackage[svgnames]{xcolor} % Udvider \color med "SVG color names"
\usepackage{graphicx}	% Kompatibilitet til visning af pixel billeder (.png, .jpg, .gif)
\usepackage{epstopdf}	% Kompatibilitet til visning af vector billeder (.eps)
\usepackage{parskip}	% Tilføjer vertikal margin til hver paragraph
\usepackage{float}		% TIllader H som positions parameter
\usepackage{subcaption}	% Tillader subfigure, subtable samt \captions
\usepackage{multirow,tabularx}
\usepackage{amssymb}	% Flere matematiske symboler
\usepackage{amsthm}      % Endnu flere matematiske symboler
\usepackage{mathtools}	% Det meste matematik (indeholder ams­math og rettelser)
\usepackage{xfrac}		% Flere fracs (\sfrac{}{})
\usepackage{listings}	% Indsæt code
\usepackage{algorithm}
\usepackage[noend]{algpseudocode}
\usepackage{fancyhdr}	% Side hoved og sidefod
\usepackage{todonotes}	% Cool to-do notes, [disable] skjuler to-do
\usepackage[bookmarks,bookmarksnumbered,hidelinks]{hyperref} % clickable pdf (til sidst)
\usepackage[bibstyle=ieee,citestyle=numeric-comp]{biblatex} % Benyt BibLaTeX til formatering

%listing settings, æøå support, font config, line number, left lines
\lstset{
    breakatwhitespace=false, breaklines=true,
    inputencoding=utf8, extendedchars=true,
    literate={å}{{\aa}}1 {æ}{{\ae}}1 {ø}{{\o}}1 {Å}{{\AA}}1 {Æ}{{\AE}}1 {Ø}{{\O}}1,
    keepspaces=true, showstringspaces=false, basicstyle=\small\ttfamily,
    frame=L, numbers=left, numberstyle=\scriptsize\color{gray},
    keywordstyle=\color{SteelBlue}\ttfamily,
    stringstyle=\color{IndianRed}\ttfamily,
    commentstyle=\color{Teal}\ttfamily,
    captionpos=b,
}

% algorithm environment
%http://tex.stackexchange.com/questions/1375/what-is-a-good-package-for-displaying-algorithms
\algnewcommand{\Let}[2]{\State #1 $\gets$ #2}
\algnewcommand{\Not}[0]{\textbf{not }}
\algnewcommand{\Implicit}[1]{\State \textit{#1}}
\algnewcommand{\LineComment}[1]{\State \(\triangleright\) #1}
\algrenewcommand\Call[2]{\textproc{#1}(#2)}
\algrenewcommand\alglinenumber[1]{{\footnotesize\color{gray}\ttfamily#1}}

\setlength{\marginparwidth}{80pt} 				% Mere brede på margin notes og to-do
\setlength{\parindent}{0cm}   					% Deaktiver afsnit indrykning
\DeclareGraphicsExtensions{.pdf,.eps,.png,.jpg,.gif}	% ændre til .png, .jpg for hurtig visning
\pagestyle{fancy}
\fancyhead[L]{Andreas Madsen - s123598}

\setcounter{tocdepth}{2}
\numberwithin{equation}{section}

\addbibresource{bibliography.bib}

\begin{document}

\title{University Timetabling \\ {\large Curriculum-based Course Timetabling using ALNS and TABU}}
\author{Andreas Madsen – s123598}
\date{April 25. 2016}

\maketitle
\tableofcontents
\pagebreak

\section{Problem Description}

The problem is very precisely defined in the project description \cite{assignment}. The problem will be restated here, but also discussed in terms of how the the delta calculations can be done.

Delta calculations is the change in the objective value when performing a simple operation. In this case only adding or removing a single course from a slot (room and time) is considered. In section \ref{sec:mutate-operations} it will be discussed how other operations can be implemented using these primitives.

The used notation is the same as in the project description \cite{assignment} and may also be found in appendix \ref{appendix:notation}.

\subsection{Constraints}
\label{sec:problem-constraints}

It will always be valid to remove a course, thus the following constraints only need to be checked when adding a course to the schedule solution.

\subsubsection{Time availability}
\begin{equation}
    \sum_{r \in R} x_{c, t, r} \le F_{c, t} \quad \forall c \in C, t \in T
\end{equation}

A course may only be assigned once in the same time slot. The dataset also contains a set of unavailable time slots for each course. For unavailable time slots $F_{c, t} = 0$ otherwise $F_{c, t} = 1$.

\textbf{Delta:} This constraint is really two separate constraints. First maintain an indicator that is true if the given course is assigned to the given time. Secondly one can use a lookup table to check if the time slot is defined unavailable.

\subsubsection{Room availability}
\begin{equation}
    \sum_{c \in C} x_{c, t, r} \le 1 \quad \forall t \in T, r \in R
\end{equation}

A slot consisting of room and time may only be used once.

\textbf{Delta:} Maintain an indicator that is true if the slot (time and room) is used.

\subsubsection{Max lectures}
\begin{equation}
\sum_{t \in T, r \in R} x_{c, t, r} \le L_c \quad \forall c \in C
\end{equation}

Prevent assigning more slots to a course than what is needed.

\textbf{Delta: } Maintain a counter of how many times a given course is assigned.

\subsubsection{Conflicting courses}
\begin{equation}
\sum_{r \in R} x_{c_1, t, r} + \sum_{r \in R} x_{c_2, t, r} \le 1 \quad \forall c_1, c_2 \in C, t \in T, \chi(c_1, c_2) = 1
\end{equation}

Conflicting courses may not be assigned the same time slot. Courses are conflicting if taught by the same lecture or part of the same curriculum.

\textbf{Delta: } Pre-calculate a list of conflicting courses for each course and maintain a list of courses assigned to each time slot. Then check that there are no conflicting courses assigned to the given time slot.

\subsection{Objective}
\label{sec:problem-objective}

There are 5 objectives which are combined to a single objective score:
\begin{equation}
\mathrm{Objective} = 10 U + 5 W + 2 A + 1 P + 1 V
\end{equation}

The delta calculations for the add and remove operations are mostly the same, just with the sign flipped. For $P$ and $W$ there are minor differences. Furthermore it can be assumed that the constrains are met, as those have already been checked.

\subsubsection{Room capacity (V)}
\begin{equation}
V = \sum_{t \in T, r \in R} V_{t, r}(x), \quad V_{t, r} = \mathrm{max}\left\{ 0, \sum_{c \in C} S_c \cdot x_{c, t, r} - C_r \right\}
\end{equation}

A course has an associated number of students $S_c$ and a room has an associated capacity $C_r$. $V$ is the penalty for exceeding this capacity.

\textbf{Delta: } One can simply lookup $S_c$ and $C_r$ for a given $(course, time, room)$ combination and calculate $V$.

\subsubsection{Unscheduled (U)}
\begin{equation}
U = \sum_{c \in C} U_c(x), \quad U_c(x) = \mathrm{max}\left\{ 0, L_c - \sum_{t \in T, r \in R} x_{c, t, r}\right\}
\end{equation}

A course has to be scheduled a given number of times $L_c$. Anything less than this is penalized.

\textbf{Delta: } Since the new solution is valid, one can simply subtract or add one to the $U$ sum.

\subsubsection{Room stability (P)}
\begin{equation}
P = \sum_{c \in C} P_c(x), \quad P_c(x) = \mathrm{max}\left\{ 0, \left|\left| \left\{ r \in R \Big\vert \sum_{t \in T} x_{c, t, r} \ge 1 \right\} \right|\right| - 1 \right\}
\end{equation}

A course should be assigned to as few different rooms as possible.

\textbf{Delta: } On addition check if $r$ is a new room for the course. On remove check if room $r$ is used only once in the current solution.

\subsubsection{Minimum working days (W)}
\begin{equation*}
W = \sum_{c \in C} W_c(x), \quad W_c(x) = \mathrm{max}\left\{ 0, M_c - \left|\left| \left\{ d \in D \Big\vert \sum_{t \in T(d), r \in R} x_{c, t, r} \ge 1 \right\} \right|\right| \right\}
\end{equation*}

A course should be assigned time slots over $M_c$ different days.

\textbf{Delta: } Count the number of days the course is spread over. On addition check if this count is less than $M_c$ and check that the course isn't already scheduled this day. On remove check that the count isn't greater than $M_c$ and that the course is scheduled only once this day.

\subsubsection{Curriculum compactness (A)}
\begin{equation*}
A = \sum_{q \in Q, t \in T} A_{q, t}(x), \quad A_{q, t}(x) = \begin{cases}
1 & \text{if} \displaystyle \sum_{c \in C(q), r \in R} x_{c, t, r} = 1 \wedge \sum_{\substack{c \in C(q), r \in R,\\ t' \in T, \Upsilon(t, t')  = 1}} x_{c, t', r} = 0 \\
0 & \text{otherwise}
\end{cases}
\end{equation*}

Penalizes gaps/holes over a day in a curriculum.

\textbf{Delta: } This is the most complicated objective in terms of delta calculation, because one also need to adjust the penalty for adjacent courses in the same curriculum. Thus if the course before was penalized then remove that penalty, similarly for the course after. If there are no courses before and after add a penalty.

\subsection{Datasets}

There are 12 datasets, each is given as 7 files:
\begin{itemize}
\item \texttt{basic.utt} - Contains basic meta information about the problem. Number of courses, rooms, days, periods pr day, etc.
\item \texttt{courses.utt} - Contains information about each course.
\item \texttt{curricula.utt} - Contains the number of courses associated to each curriculum. This file was not used, as the information could be inferred from \texttt{relation.utt}.
\item \texttt{lecturers.utt} - Contains a list of lecturers. This file was not used, as the information could be inferred from \texttt{courses.utt}.
\item \texttt{relation.utt} - Relational table that binds curriculum and course.
\item \texttt{rooms.utt} - Contains information about each room.
\item \texttt{unavailability.utt} - Contains the list of unavailable time slots for each course.
\end{itemize}

Using \texttt{basic.utt} the 13 datasets can be summarized:


\begin{table}[H]
\centering
\centerline{
\begin{tabular}{r|c c c c c c c}
	dataset & Courses & Rooms & Days & Periods per day & Curricula & Constraints & Lecturers\\ \hline
	1 & 30 & 6 & 5 & 6 & 14 & 53 & 24 \\
	2 & 82 & 16 & 5 & 5 & 70 & 513 & 71 \\
	3 & 72 & 16 & 5 & 5 & 68 & 382 & 61 \\
	4 & 79 & 18 & 5 & 5 & 57 & 396 & 70 \\
	5 & 54 & 9 & 6 & 6 & 139 & 771 & 47 \\
	6 & 108 & 18 & 5 & 5 & 70 & 632 & 87 \\
	7 & 131 & 20 & 5 & 5 & 77 & 667 & 99 \\
	8 & 86 & 18 & 5 & 5 & 61 & 478 & 76 \\
	9 & 76 & 18 & 5 & 5 & 75 & 405 & 68 \\
	10 & 115 & 18 & 5 & 5 & 67 & 694 & 88 \\
	11 & 30 & 5 & 5 & 9 & 13 & 94 & 24 \\
	12 & 88 & 11 & 6 & 6 & 150 & 1368 & 74 \\
	13 & 82 & 19 & 5 & 5 & 66 & 468 & 77
\end{tabular}}
\caption{Information from \texttt{basic.utt} for each dataset.}
\end{table}


\subsection{Choosing the metaheuristic models}

As seen the problem is highly constrained. ALNS is usually good for highly constrained, thus ALNS is an obvious choice.

The objective function is not multi objective, but do consist of many separate sub-objective functions. This suggest Evolutionary Algorithms may be a good choice, as two good solution may minimize two different sub-objectives, thus a crossover between the two solutions could minimize both and become a very good solution. However because the problem is so tightly constrained, it will be difficult to come up with a good crossover algorithm.

GRAPS is likely a poor choice, as the many constraints makes it difficult to come up with many good greedy algorithms.

The neighborhood of a given solution is at least, all possible $(course, time, room)$ additions and removal operations. However because of the big penalty (+10) of removing a $(course, time, room)$ combination, this is unlikely to ever be an immediate good choice. Thus it make sense to expand the neighborhood to include all possible moves to an available slot and all possible swaps between two $(course, time, room)$ combinations. This expanded neighborhood is huge, thus Tabu Search may also be a good choice.

\subsubsection{Conclustion}

ALNS is chosen because of the many constraints. Tabu Search is chosen because of the large neighborhood. Evolutionary Algorithms is an interesting choice and may be worth investigating, however that is out of the scope in this project.
 % Lør
\section{Metaheuristic description}
\subsection{Mutatation operations}
\label{sec:mutate-operations}

The solution schedule is defined as list of $(c \in C, t \in T, r \in R)$. Which means course $c$ is assigned to a slot given by the time $t$ and room $r$.

In this project 4 move operations are used. The operations are:
\begin{itemize}
\item \texttt{Add($c, t, r$)} - adds course $c$ to the slot $(t, r)$.  
\item \texttt{Remove($c, t, r$)} - remove course $c$ from the schedule slot $(t, r)$.
\item \texttt{Move($c, t_0, r_0, t_1, r_1$)} - moves course $c$ from $(t_0, r_0)$ to $(t_1, r_1)$.
\item \texttt{Swap($c_0, t_0, r_0, c_1, t_1, r_1$)} - course $c_0$ and course $c_1$ swap slots.
\end{itemize}

To avoid copying the solution object, it should be possible to calculate the $\Delta$ cost without actually changing the solution object. Those functions are prefixed with \texttt{Simulate}, while the functions that actually changes the solution object are prefixed with \texttt{Mutate}.

\begin{algorithm}[H]
  \caption{Add a course $c$ to slot $(t, r)$}
  \begin{algorithmic}[1]
    \Function{SimulateAdd}{$c, t, r$}
      \If{\Not \Call{ValidAdd}{$c, t, r$}} \Comment{Discussed in section \ref{sec:problem-constraints}}
          \State \Return{None}
      \EndIf
      \State \Return{\Call{CostAdd}{$c, t, r$}} \Comment{Discussed in section \ref{sec:problem-objective}}
    \EndFunction
    
    \Statex
    \Function{MutateAdd}{$c, t, r$}
        \Let{$\Delta$}{\Call{SimulateAdd}{$c, t, r$}}
        \If{$\Delta \not= None$}
            \Let{$Objective$}{$Objective + \Delta$}
            \Implicit{update data structures} \Comment{Discussed in section \ref{sec:problem-constraints} and \ref{sec:problem-objective}}
        \EndIf
    \EndFunction
  \end{algorithmic}
\end{algorithm}


\begin{algorithm}[H]
  \caption{Remove a course $c$ from slot $(t, r)$}
  \begin{algorithmic}[1]
    \Function{SimulateRemove}{$c, t, r$}
      \If{\Not \Call{ValidRemove}{$c, t, r$}} \Comment{Discussed in section \ref{sec:problem-constraints}}
          \State \Return{None}
      \EndIf
      \State \Return{\Call{CostRemove}{$c, t, r$}} \Comment{Discussed in section \ref{sec:problem-objective}}
    \EndFunction
    
    \Statex
    \Function{MutateRemove}{$c, t, r$}
        \Let{$\Delta$}{\Call{SimulateRemove}{$c, t, r$}}
        \If{$\Delta \not= None$}
            \Let{$Objective$}{$Objective + \Delta$}
            \Implicit{update data structures} \Comment{Discussed in section \ref{sec:problem-constraints} and \ref{sec:problem-objective}}
        \EndIf
    \EndFunction
  \end{algorithmic}
\end{algorithm}

Note that one typically simulates the operation to check that $\Delta < 0$ and then mutate the solution object. Thus it makes sense to add $\Delta$ as an optional argument to the mutate functions, to avoid unnecessary calculations. This was done in the implementation but is for simplicity excluded here.

The \texttt{Move} and \texttt{Swap} operations can then be implemented, using using the \texttt{Add} and \texttt{Remove} primitives. This is not the most efficient implementation as it requires mutation of the current solution, which are then reverted in the simulation case. However given the time constraints of the project, more efficient implementations wasn't made. See appendix \ref{appendix:operations} for how \texttt{Move} and \texttt{Swap} was implemented using the primitives.

\subsection{Greedy initialization}

For this particular problem a schedule with no courses scheduled is a valid solution. Thus a special initialization procedure is not necessary. However initializing using empty solution will cause ALNS and in particular Tabu Search to end up spending many iterations doing simple course additions. This is wasteful as the fixed optimization time could be spend much more productive.

Thus a simple randomized greedy initialization procedure is used to quickly go from the empty solution to a much more optimal solution.

\begin{algorithm}[H]
  \caption{Performs a greedy optimization of a solution}
  \begin{algorithmic}[1]
    \Function{GreedyInitialization}{$solution$}
      \LineComment{Create list of missing courses in random order}
      \Let{$courses$}{\Call{list}{}} \Comment{List with fast push operation}
      \ForAll{$(c, n)$ in \Call{MissingCourses}{$solution$}}
        \State \Call{Push}{c on $courses$}, $n$ times
      \EndFor
      \Let{$courses$}{\Call{Shuffle}{$courses$}}
      \State
      
      \LineComment{Create list of available slots}
      \Let{$slots$}{\Call{AvaliableSlots}{$solution$}}
      \Let{$slots$}{\Call{Shuffle}{$slots$}}
      \Let{$slots$}{\Call{Deque}{$slots$}} \Comment{Double-ended queue}
      \State
    
      \ForAll{$c$ in courses}
        \For{$i$ from $1$ to \Call{Length}{$slots$}}
          \Let{$(t, r)$}{\Call{PopRight}{$slots$}}
          \State
          \LineComment{add $(c, t, r)$ to solution if it improves the objective}
          \Let{$\Delta$}{\Call{SimulateAdd}{$c, t, r$}}
          \If{$\Delta < 0$}
            \State \Call{MutateAdd}{$c, t, r$} \Comment{Use slot}
            \State \textbf{break}
          \Else
            \State \Call{PushLeft}{$(t, r)$ on $slots$} \Comment{The slot is still avaliable}
          \EndIf
        \EndFor
      \EndFor
    \EndFunction
  \end{algorithmic}
\end{algorithm}

The algorithm tries to add all missing courses to the schedule, but it will only add the course if it improves the solution in the current situation.

\subsection{Tabu general description}

Tabu search performs local search using a neighborhood definition, it then picks the best new solution. This process is repeated in each iteration.

Doing just the local search will cause the search algorithm to each a minimum, unfortunately this minimum is very likely to only be a local minima. To escape a local minima Tabu search performs non-optimal moves once a minima is reached, this process is called diversification. This can however cause a cycling behavior, where the opposite or same moves are just applied again, thus reaching the same local minima.

Tabu search attempts to solve the cycling problem by maintaining a tabu list. The tabu list contains all temporarily illegal solutions. Once an optimal change is applied the new solution is added to the tabu list. When performing the local search the tabu list is checked before the change is attempted, this prevents one from reaching the same solution twice.

Storing all previous solutions in a tabu list is not very efficient, both in terms of space and the computational resources required for checking if two solutions are equal. Instead the neighborhood is defined by a set of moves. Once a move is applied to the solution the opposite move is added to the tabu list. By using moves one only needs to store the individual moves and check if two moves are equal, this is much more efficient.

Finally only typically also applies intensification to the tabu search algorithm. Intensification prevent the algorithm from diversifying so much, that it will takes too many iterations to find a better solution. This can be anything from take core good components from the globally best solution to just restoring the globally best solution. Usually it also involves resetting the tabu list.

\begin{algorithm}[H]
  \caption{Generalization of the Tabu search algorithm}
  \begin{algorithmic}[1]
    \Function{TabuSearch}{$solution_{init}$}
    \Let{$s_{global}$}{$solution_{init}$} \Comment{Globally best solution}
    \Let{$s_{local}$}{$solution_{init}$} \Comment{Current solution}
    \Let{$tabu$}{\Call{LimitedSet}{}} \Comment{May have infinite space}
    \State
    \Repeat
        \Let{$\Delta$, $move$}{\Call{LocalSearch}{$s_{local}$, $tabu$}} \Comment{Find best $move \not\in tabu$}
        \If{$\Delta$ < 0}
            \Let{$s_{local}$}{\Call{Apply}{$move$ on $s_{local}$}}
            \State \Call{Add}{$move$ on $tabu$}
        \Else
            \If{$s_{global}$ hasn't been updated for awhile}
                \Let{$s_{local}$}{\Call{Intensify}{$s_{global}$}} \Comment{Intensification is optional}
            \EndIf
            \Let{$s_{local}$}{\Call{Divserify}{$s_{local}$}} \Comment{Diversification is optional}
        \EndIf
        
        \If{\Call{Cost}{$s_{local}$} < \Call{Cost}{$s_{global}$}}
            \Let{$s_{global}$}{$s_{local}$}
        \EndIf
    \Until{no more time}
    \State \Return{$s_{global}$}
    \EndFunction
  \end{algorithmic}
\end{algorithm}

This algorithm have the following parameters:

\begin{table}[H]
\centering
\begin{tabular}{r|p{2cm}|p{6cm}}
	name & type & description \\ \hline
	diversification & boolean & Should diversification be used. \\
	intensification & integer & How many iterations without a globally better solution is allowed, before the intensification procedure is used. One may choose $\infty$ to disable intensification. \\
	tabu limit & integer & As new moves are added to the tabu list, old moves may be removed from the list. This parameter controls how many moves the tabu list should remember.
\end{tabular}
\caption{Parameters for generalized Tabu search}
\end{table}

\subsection{Tabu specialization}

The local search scans the neighborhood consisting of:
\begin{itemize}
\item All possible additions of a course to an available slot.
\item All possible removals of a $(course, time, room)$ combination from the schedule.
\item All possible moves of an existing $(course, time, room)$ combination to an available slot.
\item All possible swaps between two $(course, time, room)$ combinations.
\end{itemize}

The swap operation is in particularly expensive, on a typical run the tabu search will spend $0.6$ to $0.8$ seconds pr. iteration, without using the swap operation. Adding the swap operation adds an additional $2.5$ to $3.3$ seconds pr. iterations. Adding swap moves to the neighborhood thus introduces a tradeoff between having a large neighborhood and performing many iterations. To optimize this tradeoff a parameter was added.

Diversification is done by removing random $(course, time, room)$ combinations from the schedule. How may there should be removed is controlled by a parameter.

Intensification is done by simply restoring the globally best solution and resetting the tabu lists.

The tabu search have just a single tabu list. However because the neighborhood is expressed using different move operations, individual tabu lists where used. This simply because of the project time constraints, joining the tabu lists is completely possible.

\begin{table}[H]
\centering
\begin{tabular}{r|p{2cm}|p{6cm}}
	name & type & description \\ \hline
	diversification & integer & How many $(course, time, room)$ combinations should be removed. May be zero to disable diversification. \\
	intensification & integer & Same as in the generalized tabu search. \\
	tabu limit & integer & This parameter controls the tabu limit of all the tabu lists. \\
	allow swap & $\{$always, dynamic, never$\}$ & If \textit{always} the swap neighborhood is always checked. If \textit{never} the swap neighborhood is never checked. Additionally if \textit{dynamic}, the swap neighborhood is only checked if none of the other operations could reduce the objective.
\end{tabular}
\caption{Parameters for specialized Tabu search}
\end{table}



\subsection{ALNS general description}

LNS is an search algorithm that uses a destroy method and a repair method. The destroy method will remove a part of the current solution. The destroy method in it self is unlikely to improve the objective value, thus it is followed by a repair method that will search for a better solution. The destroy and repair methods are typically stochastic.

The idea is that by removing parts of the solution without expecting a better solution, the search neighborhood becomes very large without being computationally expensive. This will of cause require that the repair method to do more good than the destroy method does harm, at least on average. However the performance of the destroy and search method will most likely depend on the specific dataset. ALNS generalizes LNS such that more than one destroy and one repair algorithm can be used. ALNS will then dynamically choice the best destroy and repair algorithms for the specific dataset.

ALNS choices the repair and destroy method by updating the probability of selecting the different repair and destroy methods. In each iteration this probability is used to randomly select the method.

The probability update equations requires the following model parameters:

\begin{table}[H]
\centering
\begin{tabular}{r|p{2.5cm}|p{6cm}}
	name & type & description \\ \hline
	$\lambda$ & ratio $\in [0, 1]$ & remember parameter used in the moving average update of the probabilities. \\
	$w_{global}$ & positive integer & reward for a globally better solution \\
	$w_{current}$ & positive integer & reword for a locally better solution \\
	$w_{accept}$ & positive integer & reward for accepting the new solution \\
	$w_{reject}$ & positive integer & \textit{reward} for rejecting the new solution
\end{tabular}
\caption{Parameters for generalized ALNS search}
\end{table}

First the reward is calculated as $\Psi = \max\{w_{global}, w_{current}, w_{accept}, w_{reject}\}$ where the $w$ values are zero if the corresponding reward condition weren't met. If the selected destroy method has index $d$ and the repair method index $r$, the preference values are then updated as:
\begin{align}
p_d^- = \lambda p_d^- + (1 - \lambda)\Psi \\
p_r^+ = \lambda p_r^+ + (1 - \lambda)\Psi
\end{align}

Here $p^-$ are the preference values for destroy and $p^+$ are for repair.

To convert the preference values to actual probabilities, simply scale by the sum:
\begin{align}
\phi_d^- = \frac{p_d^-}{\sum_{i} p_i^-} \\
\phi_r^+ = \frac{p_r^+}{\sum_{i} p_i^+}
\end{align}

To sample from these distributions, can use compare the commutative sum with a uniformly random number between 0 and 1.

\begin{algorithm}[H]
  \caption{Samples using the preference values $p$}
  \begin{algorithmic}[1]
    \Function{SampleFunction}{$p$}
      \Let{$\phi$}{\Call{Scale}{$p$}}
      \Let{c}{\Call{CumSum}{$\phi$}}
      \Let{rand}{\Call{UniformRandom}{0, 1}}
      \State \Return{\Call{Bisect}{c, rand}} \Comment{Finds first index where $rand < c_i$}
    \EndFunction
  \end{algorithmic}
\end{algorithm}

The entire algorithm can now be stated as:

\begin{algorithm}[H]
  \caption{Generalization of the ALNS search algorithm}
  \begin{algorithmic}[1]
    \Function{AlnsSearch}{$solution_{init}$}
      \Let{$s_{global}$}{$solution_{init}$} \Comment{Globally best solution}
      \Let{$s_{local}$}{$solution_{init}$} \Comment{Current solution}
      \Let{$p^+$, $p^-$}{vector of 1s}
      \State
      \Repeat
      \Let{$d$}{\Call{SampleFunction}{$p^-$}}
      \Let{$r$}{\Call{SampleFunction}{$p^+$}}
      \Let{$s_{local}$}{\Call{Repair}{\Call{Destroy}{$s_{local}, d$}, $r$}}
      \State
      \Let{$\Psi$}{$\max\{w_{global}, w_{current}, w_{accept}, w_{reject}\}$}
      \Let{$p_d^-$}{$\lambda p_d^- + (1 - \lambda)\Psi$}
      \Let{$p_d^+$}{$\lambda p_d^+ + (1 - \lambda)\Psi$}
      \State
      \If{\Call{Cost}{$s_{local}$} < \Call{Cost}{$s_{global}$}}
        \Let{$s_{global}$}{$s_{local}$}
      \EndIf
      \Until{no more time}
      \State \Return{$s_{global}$}
    \EndFunction
  \end{algorithmic}
\end{algorithm}

\subsection{ALNS specialization}

The problem specific ALNS implementation is almost identical to the generalized ALNS. The destroy methods remove (course, time, room) combinations according to specific rules, and the repair methods adds missing courses to the schedule. Because removing a course is always valid and adding courses can be validated continuously, the $w_{accept}$ and $w_{reject}$ parameters serves no purpose, thus the gain simply becomes:
\begin{equation}
\Psi = \max\{w_{global}, w_{current}\}
\end{equation}

The parameters are:

\begin{table}[H]
\centering
\begin{tabular}{r|p{2.5cm}|p{6cm}}
	name & type & description \\ \hline
	$\lambda$ & ratio $\in [0, 1]$ & remember parameter used in the moving average update of the probabilities. \\
	$w_{global}$ & positive integer & reward for a globally better solution \\
	$w_{current}$ & positive integer & reword for a locally better solution \\
	$remove$ & positive integer & number of courses removed in each destroy function
\end{tabular}
\caption{Parameters for generalized ALNS search}
\end{table}

\subsubsection{Destroy functions}

There are 4 destroy functions, they all remove a given number ($remove$) of courses using some strategy.

\begin{algorithm}[H]
  \caption{remove random (course, time, room) combinations from the solution}
  \begin{algorithmic}[1]
    \Function{DestroyFullyRandom}{$solution$}
      \For{$(c, t, r)$ in \Call{UniformSample}{$\{(c, t, r)\}$, $remove$}}
        \State \Call{MutateRemove}{$c, t, r$}
      \EndFor
    \EndFunction
  \end{algorithmic}
\end{algorithm}

\begin{algorithm}[H]
  \caption{remove random (course, time, room) combinations from a curriculum}
  \begin{algorithmic}[1]
    \Function{DestroyCurriculum}{$solution$}
      \Let{$q$}{\Call{UniformSample}{$Q, 1$}}
      \For{$(c, t, r)$ in \Call{UniformSample}{$\{(c, t, r)\ |\ c \in C(q)\}$, $remove$}}
        \State \Call{MutateRemove}{$c, t, r$}
      \EndFor
    \EndFunction
  \end{algorithmic}
\end{algorithm}

\begin{algorithm}[H]
  \caption{remove random (course, time, room) combinations from a day}
  \begin{algorithmic}[1]
    \Function{DestroyDay}{$solution$}
      \Let{$d$}{\Call{UniformSample}{$D, 1$}}
      \For{$(c, t, r)$ in \Call{UniformSample}{$\{(c, t, r)\ |\ t \in d\}$, $remove$}}
        \State \Call{MutateRemove}{$c, t, r$}
      \EndFor
    \EndFunction
  \end{algorithmic}
\end{algorithm}

\begin{algorithm}[H]
  \caption{remove random (course, time, room) combinations where the course is fixed}
  \begin{algorithmic}[1]
    \Function{DestroyCourse}{$solution$}
      \Let{$c_d$}{\Call{UniformSample}{$C, 1$}}
      \For{$(c, t, r)$ in \Call{UniformSample}{$\{(c, t, r)\ |\ c = c_d\}$, $remove$}}
        \State \Call{MutateRemove}{$c, t, r$}
      \EndFor
    \EndFunction
  \end{algorithmic}
\end{algorithm}

\subsubsection{Repair functions}

The repair methods attempt to insert all missing courses.

\begin{algorithm}[H]
  \caption{picks the first slot for a course that has $\Delta < 0$}
  \begin{algorithmic}[1]
    \Function{VeryGreedyRepair}{$solution$}
      \ForAll{$(c, missing)$ in \Call{MissingCourses}{$solution$}}
        \ForAll{$(t, r)$ in \Call{AvaliableSlots}{$solution$}}
          \Let{$\Delta$}{\Call{SimulateAdd}{$c, t, r$}}
          \If{$\Delta < 0$}
            \State \Call{MutateAdd}{$c, t, r$}
            \Let{$missing$}{$missing - 1$}
            \If{$missing = 0$} \textbf{break} \EndIf
          \EndIf
        \EndFor
      \EndFor
    \EndFunction
  \end{algorithmic}
\end{algorithm}

\begin{algorithm}[H]
  \caption{evaluate $\Delta$ independently and takes the best for each course}
  \begin{algorithmic}[1]
    \Function{BestPlacementRepair}{$solution$}
      \ForAll{$(c, missing)$ in \Call{MissingCourses}{$solution$}}
        \LineComment{Find the best $missing$ slots assuming independent $\Delta$}
        \Let{$slots$}{\Call{AvaliableSlots}{$solution$}}
        \Let{$best$}{\Call{MinSort}{$slots$ by \Call{SimulateAdd}{$c, t, r$}, $missing$}}
        \State
        \ForAll{$(t, r)$ in $best$}
          \Let{$\Delta$}{\Call{SimulateAdd}{$c, t, r$}} \Comment{revalidate improvement}
          \If{$\Delta < 0$}
            \State \Call{MutateAdd}{$c, t, r$}
          \EndIf
        \EndFor
      \EndFor
    \EndFunction
  \end{algorithmic}
\end{algorithm}

An issue with the chosen repair methods is that \texttt{BestPlacementRepair} will almost always perform better than \texttt{VeryGreedyRepair}. However it also require much more computation time. ALNS does not penalize computation time, thus it will likely often choose \texttt{BestPlacementRepair} even if \texttt{VeryGreedyRepair} was better because it allowed more iterations.

Also note that the repair methods aren't random. This was done for implementation simplicity. It also improves speed as there no need to generate the full list of missing courses and available slots. Because the destroy methods are random, it is unlikely that the lack of randomness is a big issue. However given more time this would be worth exploring.

 % Søn
\section{Parameter tuning}

In order to find the best set of parameters for the ALNS and Tabu search, different parameter combinations was tried (see section \ref{sec:parameter-tabu} and \ref{sec:parameter-alns}). Each parameter combination was tried 3 times using different initializations.

Because the problems aren't equally difficult and because the objective value aren't normalized, the objective value for each dataset can't be directly compared. To accommodate the best objective value for each dataset is used to normalize the objective (percentage gap):
\begin{equation}
\tilde{z}_i = \frac{z_i - z^*}{z^*}
\end{equation}
Here $z_i$ is the objective value and $z^*$ is the best objective value for the dataset.

Because one wishes to avoid overfitting of the parameters, a subset of the entire dataset is chosen for parameter optimization, this is called the training dataset. As there do not appear to be any pattern in the dataset id, all odd dataset are chosen for parameter optimization.

\begin{table}[H]
\centering
\begin{tabular}{l|rrrrrrr}
 dataset id &   1 &   3 &   5 &    7 &   9 &   11 &   13 \\
\hline
 Tabu   &  35 & 706 & 896 & 1390 & 765 &   36 &  794 \\
 ALNS   &  24 & 211 & 761 &  211 & 200 &    5 &  166 \\
 both   &  24 & 211 & 761 &  211 & 200 &    5 &  166 \\
\end{tabular}
\caption{Best objective value for each training dataset}
\end{table}

\subsection{Tabu}
\label{sec:parameter-tabu}

\begin{table}[H]
\centering
\centerline{\begin{tabular}{rr|ccc}
 &  & \multicolumn{3}{c}{\texttt{intensification}}\\
 &  & 2 & 10 & None\\
\hline
\multirow{3}{*}{\texttt{diversification}} & None & (4.90, 0.52) & (5.36, 0.37) & (5.53, 0.71)\\
 & 1 & (5.07, 0.97) & (5.34, 0.17) & (5.29, 0.63)\\
 & 5 & (5.19, 0.41) & (4.14, 0.12) & (5.06, 0.77)\\
\end{tabular}}
\caption{Shows $(\mu, \sigma)$ with \texttt{allow\_swap=dynamic} and \texttt{tabu\_limit=40} fixed}
\end{table}

\begin{table}[H]
\centering
\centerline{\begin{tabular}{rr|cccc}
 &  & \multicolumn{4}{c}{\texttt{tabu\_limit}}\\
 &  & 10 & 20 & 40 & None\\
\hline
\multirow{3}{*}{\texttt{allow\_swap}} & never & (5.53, 0.70) & (5.91, 0.89) & (6.52, 0.60) & (5.93, 0.43)\\
 & always & (9.12, 0.18) & (8.77, 0.26) & (9.16, 0.87) & (8.82, 0.44)\\
 & dynamic & (5.52, 0.69) & (5.10, 0.59) & (4.14, 0.12) & (5.45, 0.27)\\
\end{tabular}}
\caption{Shows $(\mu, \sigma)$ with \texttt{diversification=5} and \texttt{intensification=10} fixed}
\end{table}

\begin{table}[H]
\centering
\begin{tabular}{r|c}
parameter & value \\ \hline
allow swap & dynamic \\
tabu limit & 40 \\
intensification & 10 \\
diversification & 5
\end{tabular}
\caption{Best Tabu search parameters with $\mu = 4.139$ and $\sigma = 0.122$}
\end{table}

\subsection{ALNS}
\label{sec:parameter-alns}

\begin{table}[H]
\centering
\centerline{\begin{tabular}{rr|ccc}
 &  & \multicolumn{3}{c}{\texttt{remove}}\\
 &  & 1 & 3 & 5\\
\hline
\multirow{3}{*}{\texttt{update\_lambda}} & 0.9 & (0.50, 0.10) & (0.59, 0.01) & (0.82, 0.05)\\
 & 0.95 & (0.30, 0.08) & (1.07, 0.03) & (1.02, 0.12)\\
 & 0.99 & (0.56, 0.09) & (1.69, 0.11) & (1.52, 0.07)\\
\end{tabular}}
\caption{Shows $(\mu, \sigma)$ with \texttt{w\_global=10} and \texttt{w\_current=5} fixed}
\end{table}

\begin{table}[H]
\centering
\centerline{\begin{tabular}{rr|ccc}
 &  & \multicolumn{3}{c}{\texttt{w\_current}}\\
 &  & 1 & 3 & 5\\
\hline
\multirow{2}{*}{\texttt{w\_global}} & 5 & (0.55, 0.06) & (0.52, 0.12) & (0.44, 0.03)\\
 & 10 & (0.49, 0.04) & (0.32, 0.10) & (0.30, 0.08)\\
\end{tabular}}
\caption{Shows $(\mu, \sigma)$ with \texttt{update\_lambda=0.95} and \texttt{remove=1} fixed}
\end{table}

\begin{table}[H]
\centering
\begin{tabular}{r|c}
parameter & value \\ \hline
$\lambda$ & 0.95 \\
$w_{global}$ & 10 \\
$w_{current}$ & 5 \\
remove & 1 \\
\end{tabular}
\caption{Best ALNS parameters with $\mu = 0.3024$ and $\sigma = 0.0846$}
\end{table} % Man
\section{Test results}

The best parameters form the parameter tuning in section \ref{sec:parameter-tuning}, are used to test both the Tabu and ALNS model. All odd numbered datasets was used for parameter tuning, thus those datasets shouldn't be used for testing. But for completeness the search algorithms was reapplied on the training set. This time the parameters are fixed thus the number of runs can be increased to 5 runs per dataset.

\begin{table}[H]
\centering
\centerline{\begin{tabular}{rr|cc}
& & Tabu & ALNS \\
\hline
\multirow{7}{*}{train} & 1 & (3.00, 0.84) & (0.22, 0.18) \\
& 3 & (3.07, 0.08) & (0.23, 0.15) \\
& 5 & (0.14, 0.04) & (0.07, 0.05) \\
& 7 & (6.33, 0.22) & (0.20, 0.11) \\
& 9 & (3.19, 0.24) & (0.13, 0.09) \\
& 11 & (12.80, 2.45) & (0.80, 0.81) \\
& 13 & (3.90, 0.17) & (0.11, 0.11) \\
\hline
\multirow{6}{*}{test} & 2 & (3.17, 0.27) & (0.09, 0.11) \\
& 4 & (4.88, 0.35) & (0.09, 0.08) \\
& 6 & (5.06, 0.35) & (0.05, 0.05) \\
& 8 & (5.19, 0.43) & (0.12, 0.07) \\
& 10 & (6.22, 0.39) & (0.14, 0.07) \\
& 12 & (0.92, 0.10) & (0.06, 0.06) \\
\hline
\multicolumn{2}{c|}{all train} & (4.63, 0.46) & (0.25, 0.14) \\
\multicolumn{2}{c|}{all test} & (4.24, 0.15) & (0.09, 0.04) \\
\end{tabular}}
\caption{Test and train results over 5 runs using best parameters}
\end{table}

The Tabu mean is a bit too high given the standard deviance calculated in section \ref{sec:parameter-tuning}. This appears to be caused by dataset number 11, where the Tabu search performs extremely poorly. But overall the values are close to what one would expect, from the parameter tuning in section \ref{sec:parameter-tuning}. The test objective values are actually surprisingly good, as they are a little smaller than those produced by the train datasets.
 % Man, Fre
\section{Conclusion}

ALNS consistently outperforms Tabu search. As both algorithms reaches a point where it becomes hard to find a much better solution, this must be because ALNS covers a much boarder solution space.

Diversification in Tabu search is meant to move the search to another part of the solution space. But clearly this is not accomplished sufficiently. This is likely because Tabu search spends a lot of time just in validating solutions, this is particularly the case with the swap operation. ALNS don't do this, as it just looks at the best placement for missing courses. Improving the performance of the move and swap operations could make a huge difference for Tabu search. The fact that the dynamic neighborhood consistently outperforms any other neighborhood credits this hypothesis too.

Even if the move and swap operations where made faster, it is possible that the complexity of the problem makes it impossible for those operations to become sufficiently fast. Using a dynamic neighborhood thus still makes sense and in particularly this is something that could be applied to many other problems. It has definitely been shown to make a big improvement.

In parameter tuning for ALNS it was found that using the same $w_{current}$ and $w_{global}$ gave the best results. This is something that is not intuitive at first but as discussed indicates that the a globally better solution is found by a long line of good repair and destroy choices. If one did not have the computational resources to optimize these parameters, setting them to the same value may be a good choice.

To improve the ALNS method further one could analyze the repair and destroy method selection probabilities to learn about what works and more importantly doesn't work for some datasets. Using this knowledge more fine tuned algorithms could be constructed. However one also have to be carful not to do such must fine tuning that it becomes overfitting.

At last one could explore combining the methods. A simple way of doing this, could be to initialize the Tabu search using the ALNS solution. This would ensure that the ALNS solution is in a local minima and there isn't some nearby better solution, that can be found by simple diversification.
 % Fre

\pagebreak
\appendix
\section{Notaton}
\label{appendix:notation}

Notation and text descriptions are from the project description \cite{assignment}.

\subsubsection*{Sets}

\begin{description}
\item[$C$] The set of courses
\item[$L$] The set of lecturers
\item[$R$] The set of rooms
\item[$Q$] The set of curricula
\item[$T$] The set of time slots. i.e. all pairs of days and periods
\item[$D$] The set of days
\item[$T(d)$] The set of time slots that belongs to day $d \in D$
\item[$C(q)$] The set of courses that belongs to curriculum $q \in Q$
\end{description}

\subsubsection*{Parameters}

\begin{description}
\item[$L_c$] The number of lectures there should be for course $c \in C$
\item[$C_r$] The capacity of room $r \in R$
\item[$S_c$] The number of students attending course $c$
\item[$M_c$] The minimum number of days that course c should be spread across
\end{description}

\begin{align*}
F_{c,t} &= \begin{cases}
1 & \text{if course $c \in C$ is available in time slot $t \in T$} \\
0 & \text{otherwise}
\end{cases} \\
\chi(c_1, c_2) &= \begin{cases}
1 & \parbox[t]{.8\textwidth}{if course $c_1 \in C$ is different from course $c_2 \in C$ ($c_1 \not = c_2$) and conflicting, ie. are taught by the same lecturer or are part of the same curriculum.} \\
0 & \text{otherwise}
\end{cases} \\
\Upsilon(t_1, t_2) &= \begin{cases}
1 & \parbox[t]{.8\textwidth}{if time slot $t_1$ and $t_2$ belongs to the same day and are adjacent to each other} \\
0 & \text{otherwise}
\end{cases}
\end{align*}

\subsubsection*{Decision Variables}

\begin{equation*}
x_{c, t, r} = \begin{cases}
1 & \text{if class $c \in C$ is allocated to room $r \in R$ in time slot $t\in T$} \\
0 & \text{otherwise}
\end{cases}
\end{equation*}

\section{Move and swap operation}
\label{appendix:operations}


\begin{algorithm}[H]
  \caption{Move a course $c$ from slot $(t_0, r_0)$ to $(t_1, t_2)$}
  \begin{algorithmic}[1]
    \Function{SimulateMove}{$c, t_0, r_0, t_1, r_1$}
      \Let{$\Delta_{remove}$}{\Call{SimulateRemove}{$c, t_0, r_0$}}
      \If{$\Delta_{remove} = None$}
          \State \Return{None}
      \EndIf
      \State \Call{MutateRemove}{$c, t_0, r_0$}
      \State
      \Let{$\Delta_{add}$}{\Call{SimulateAdd}{$c, t_1, r_1$}}
      \If{$\Delta_{add} = None$}
          \State \Call{MutateAdd}{$c, t_0, r_0$} \Comment{Revert remove operation}
          \State \Return{None}
      \EndIf
      \State
      \State \Call{MutateAdd}{$c, t_0, r_0$} \Comment{Revert remove operation}
      \State
      \State\Return{$\Delta_{remove} + \Delta_{add}$}
    \EndFunction
    
    \Statex
    \Function{MutateMove}{$c, t_0, r_0, t_1, r_1$}
        \Let{$\Delta$}{\Call{SimulateMove}{$c, t_0, r_0, t_1, r_1$}}
        \If{$\Delta \not= None$}
            \State\Call{MutateRemove}{$c, t_0, r_0$}
            \State\Call{MutateAdd}{$c, t_1, r_1$}
        \EndIf
    \EndFunction
  \end{algorithmic}
\end{algorithm}


\begin{algorithm}[H]
  \caption{Course $c_0$ and course $c_1$ swaps slots.}
  \begin{algorithmic}[1]
    \Function{SimulateSwap}{$c_0, t_0, r_0, c_1, t_1, r_1$}
      \Let{$\Delta_{remove, 0}$}{\Call{SimulateRemove}{$c_0, t_0, r_0$}}
      \If{$\Delta_{remove, 0} = None$}
          \State \Return{None}
      \EndIf
      \State \Call{MutateRemove}{$c_0, t_0, r_0$}
      \State
      \Let{$\Delta_{remove, 1}$}{\Call{SimulateRemove}{$c_1, t_1, r_1$}}
      \If{$\Delta_{remove, 1} = None$}
          \State \Call{MutateAdd}{$c_0, t_0, r_0$} \Comment{Revert remove operation}
          \State \Return{None}
      \EndIf
      \State \Call{MutateRemove}{$c_1, t_1, r_1$}
      \State
      \Let{$\Delta_{add, 0}$}{\Call{SimulateAdd}{$c_0, t_1, r_1$}}
      \If{$\Delta_{add, 0} = None$}
          \State \Call{MutateAdd}{$c_1, t_1, r_1$} \Comment{Revert operations}
          \State \Call{MutateAdd}{$c_0, t_0, r_0$}
          \State \Return{None}
      \EndIf
      \State \Call{MutateAdd}{$c_0, t_1, r_1$}
      \State
      \Let{$\Delta_{add, 1}$}{\Call{SimulateAdd}{$c_1, t_0, r_0$}}
      \If{$\Delta_{add, 1} = None$}
          \State \Call{MutateRemove}{$c_0, t_1, r_1$} \Comment{Revert operations}
          \State \Call{MutateAdd}{$c_1, t_1, r_1$}
          \State \Call{MutateAdd}{$c_0, t_0, r_0$}
          \State \Return{None}
      \EndIf
      \State
      \State \Call{MutateRemove}{$c_0, t_1, r_1$} \Comment{Revert operations}
      \State \Call{MutateAdd}{$c_1, t_1, r_1$}
      \State \Call{MutateAdd}{$c_0, t_0, r_0$}
      \State
      \State \Return{$\Delta_{remove, 0} + \Delta_{remove, 1} + \Delta_{add, 0} + \Delta_{add, 1}$}
    \EndFunction
    
    \Statex
    \Function{MutateSwap}{$c_0, t_0, r_0, c_1, t_1, r_1$}
        \Let{$\Delta$}{\Call{SimulateSwap}{$c_0, t_0, r_0, c_1, t_1, r_1$}}
        \If{$\Delta \not= None$}
            \State \Call{MutateRemove}{$c_0, t_0, r_0$}
            \State \Call{MutateRemove}{$c_1, t_1, r_1$}
            \State \Call{MutateAdd}{$c_0, t_1, r_1$}
            \State \Call{MutateAdd}{$c_1, t_0, r_0$}
        \EndIf
    \EndFunction
  \end{algorithmic}
\end{algorithm}


\pagebreak
\printbibliography

\end{document}
