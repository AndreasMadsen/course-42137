\section{Introduction}

The curriculum-based university timetabling problem is about assigning courses, which are associated to a set of curriculums, to a schedule. Each slot in this schedule consists of a room and a time, where the time is discretized into a given number of days and periods per day. Each course are to be scheduled a given number of times.

The curriculum part of the problem, is about making sure a student who attend a given curriculum can take all courses, meaning there are no overlapping courses within the  curriculum. There is also an optimization component, which prefers schedule where there are no holes for a given curriculum during a day (called curriculum compactness).

For solving this problem a meta heuristic approach is taken. Tabu Search and ALNS will both be used to solve the problem independently and then be compared.